% !TEX TS-program = pdflatex
% !TEX encoding = UTF-8 Unicode

% This is a simple template for a LaTeX document using the "article" class.
% See "book", "report", "letter" for other types of document.

\documentclass[14pt]{article} % use larger type; default would be 10pt

\usepackage[utf8]{inputenc} % set input encoding (not needed with XeLaTeX)

%%% Examples of Article customizations
% These packages are optional, depending whether you want the features they provide.
% See the LaTeX Companion or other references for full information.

%%% PAGE DIMENSIONS
\usepackage{geometry} % to change the page dimensions
\geometry{a4paper} % or letterpaper (US) or a5paper or....
% \geometry{margin=2in} % for example, change the margins to 2 inches all round
% \geometry{landscape} % set up the page for landscape
%   read geometry.pdf for detailed page layout information

\usepackage{graphicx} % support the \includegraphics command and options

%support math
\usepackage{amsthm}

% \usepackage[parfill]{parskip} % Activate to begin paragraphs with an empty line rather than an indent

%%% PACKAGES
\usepackage{booktabs} % for much better looking tables
\usepackage{array} % for better arrays (eg matrices) in maths
\usepackage{paralist} % very flexible & customisable lists (eg. enumerate/itemize, etc.)
\usepackage{verbatim} % adds environment for commenting out blocks of text & for better verbatim
\usepackage{subfig} % make it possible to include more than one captioned figure/table in a single float
% These packages are all incorporated in the memoir class to one degree or another...

%%% HEADERS & FOOTERS
\usepackage{fancyhdr} % This should be set AFTER setting up the page geometry
\pagestyle{fancy} % options: empty , plain , fancy
\renewcommand{\headrulewidth}{0pt} % customise the layout...
\lhead{}\chead{}\rhead{}
\lfoot{}\cfoot{\thepage}\rfoot{}

%%% SECTION TITLE APPEARANCE
\usepackage{sectsty}
\allsectionsfont{\sffamily\mdseries\upshape} % (See the fntguide.pdf for font help)
% (This matches ConTeXt defaults)

%%% ToC (table of contents) APPEARANCE
\usepackage[nottoc,notlof,notlot]{tocbibind} % Put the bibliography in the ToC
\usepackage[titles,subfigure]{tocloft} % Alter the style of the Table of Contents
\renewcommand{\cftsecfont}{\rmfamily\mdseries\upshape}
\renewcommand{\cftsecpagefont}{\rmfamily\mdseries\upshape} % No bold!
\def\stir#1#2{\left\{
\begin{array}{c}
#1\\#2
\end{array}
\right\}}
%%% END Article customizations

%%% The "real" document content comes below...

\title{Bell Number}
\author{Song Yangyu}
%\date{} % Activate to display a given date or no date (if empty),
         % otherwise the current date is printed 

\begin{document}
\maketitle

\section{Task Description}
\subsection{Back Ground Information}
For a $n$ elements, the number of ways to partition this it into $k$ non-empty, non-overlapping subsets
is called Stirling Number, denoted by: 
$$
\stir{n}{k}
$$
For example, \\*
\begin{center}
$\stir{3}{2} = 2$, because we can divide it into $\{\{1\},\{2,3\}\}, \{\{1,2\},\{3\}\},\{\{1,3\},\{2\}\}  $,\\*
$\stir{3}{1} = 1 $, because we can divide it into$\{1,2,3\} = \{\{1\},\{2\},\{3\}\} $.
\end{center}
and we define:
$$\stir{0}{0} = 1, \stir{n}{0} = 0, (n > 0)$$
Because there's always a way to divide 0 element into 0 subsite by doing nothing, and there's no way to divide non-0 number of elements into 0 subsite.\\
You may notice that here's very nice priority of Stirling Numbers, like:
$$\stir{n}{2} = 2^{n-1} - 1$$
\\
The Bell Number of $n$ is the sum of all the stirling numbers of sets with n elements, i.e.,
$$B(n) = \sum_{k=0}^{n}\stir{n}{k}$$
\section{Main Task}
Now we're interested in finding the bell number of given input $n$.
\section{Input}
First a numer T would be given as the number of test cases, then follows T numbers, each number represents the number $n$.
$$
n <= 10^4, T <= 100
$$
\section{Output}
For the given number n, output the bell number of the that n. each number per line.\\
Since the bell number for larger n would be huge, modulo the result by 100000007.

\section{Sample IO}
\subsection*{Input}
4\\*
0\\*
2\\*
4\\*
10000\\
\subsection*{Output}
1\\*
2\\*
15\\*
22785804\\*

\end{document}
